\hypertarget{index_Introduction}{}\section{Introduction}\label{index_Introduction}
Thunder Calendar a été réalisé par Valentin Feld et Antoine Jeannot en Printemps 2015.

Cette application de calendrier est le fruit de l\textquotesingle{}apprentissage du C++ et de ses design patterns réalisé en L\+O21 à l\textquotesingle{}Université de Technologie de Compiègne.

En dépit de lourdes contraintes de temps dûes aux délais impartis, des concessions ont dûes être faites.

Nous avons essayé de livrer le programme le plus complet possible tout en respectant un maximum le cahier des charges imposé.

Aussi, de nombreuses optimisations sont possibles et réalisables grâce à une architecture qui se veut évolutive et maintenable.

La conception de ce projet nous aura permis d\textquotesingle{}améliorer notre travail d\textquotesingle{}équipe ainsi que nos compétences manageriales, sous la pression des examens et des rendus de nos cinq autres projets simultanés.

Elle a également été l\textquotesingle{}occasion de se former aux outils de collaboration en ligne comme Git\+Hub.

De plus, les méthodes S\+C\+R\+U\+M ont été employées dès le début de l\textquotesingle{}analyse du problème, de la modélisation logique des données à la création de cette documentation.

Vous trouverez ci-\/dessous\+:
\begin{DoxyItemize}
\item l\textquotesingle{}executable
\item la video de présentation
\item les impressions d\textquotesingle{}écran
\item la modélisation U\+M\+L
\item le lien vers les fichiers sources disponibles sur Git\+Hub
\end{DoxyItemize}

Pour nous contacter\+:
\begin{DoxyItemize}
\item Antoine\+: \href{mailto:antoine.jeannot@etu.utc.fr}{\tt antoine.\+jeannot@etu.\+utc.\+fr}
\item Valentin\+: \href{mailto:valentin.feld@etu.utc.fr}{\tt valentin.\+feld@etu.\+utc.\+fr} 
\end{DoxyItemize}